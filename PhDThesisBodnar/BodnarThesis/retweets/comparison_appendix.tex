\chapter{A Comparison of Three Types of Message Propagation Models}
\label{appendix:comparetypes}
Here we repeat the analysis in chapter \ref{retweets} on the H7N9 dataset using three different types of propagation models: Retweets (Retweets), Retweets+Hidden flow (Combined), Retweets+Hidden Flow - Spam (No Spam). There are 158364, 139959, and 140174 independent, original messages detected in the models, respectively. 


Due to the large data sizes, we decide to \emph{not} present the statistical significance between the different models as it may be deceptive to the reader. For example, we use the No Spam model in the main paper, which can be predicted using follower count with a correlation of 0.4169. Given the size of the data, the Retweet dataset is different in a statistically significant sense if the correlation is outsize of the range \(.4109 \leq r \leq .4229\) (Fisher z-transformation, two-tailed z-test), approximately a 1.4\% change, which is unlikely to be operationally different.

Here, we repeat the steps described in chapter \ref{retweets} by replicating the modeling of followers, user type and emotions on retweet rates (see table \ref{tab:modelcor3}) and perform keyword model selection (see table \ref{tab:keywordregressionsupplemental}). Instead of doing analysis on an 85\%, 10\%, 5\% split as described in section \ref{tab:combinedfeatures}, we simply work with the full datasets. Thus we do not include the aggregated models, as it would be deceptive to present the fits without a hold out dataset. Note that in 3 out of 4 models, the Combined dataset (the one without spam Tweet copies removed) is the most poorly fit. This loss of preventiveness may be due to a random selection of what Tweets to repost by a simple spam bot compared to the other two datasets, where a human provides some selective pressure in what he or she chooses to retweet.


\begin{table}[]
\centering
\begin{tabular}{|l|r|r|r|r|}
\hline
\multicolumn{1}{|c|}{Model}           & \multicolumn{1}{c|}{Min-N} & \multicolumn{1}{c|}{Retweets} & \multicolumn{1}{c|}{Combined} & \multicolumn{1}{c|}{No Spam} \\ \hline
\multirow{2}{*}{SVMR}                 & 100                        & 0.0132                        & 0.0059                        & -0.0029                      \\ \cline{2-5} 
                                      & 1000                       & -0.0007                       & 0.0013                        & -0.0217                      \\ \hline
\multirow{2}{*}{Regression Tree (2)}  & 100                        & 0.0688                        & 0.0497                        & 0.0747                       \\ \cline{2-5} 
                                      & 1000                       & 0.0653                        & 0.0475                        & 0.0637                       \\ \hline
\multirow{2}{*}{Regression Tree (5)}  & 100                        & 0.0829                        & 0.0719                        & 0.0962                       \\ \cline{2-5} 
                                      & 1000                       & 0.1033                        & 0.0742                        & 0.0927                       \\ \hline
\multirow{2}{*}{Regression Tree (10)} & 100                        & 0.0707                        & 0.0742                        & 0.0642                       \\ \cline{2-5} 
                                      & 1000                       & 0.0891                        & 0.0960                        & 0.1066                       \\ \hline
\multirow{2}{*}{Gradient Boosting}    & 100                        & 0.1552                        & 0.1473                        & 0.1719                       \\ \cline{2-5} 
                                      & 1000                       & 0.1605                        & 0.1408                        & 0.1600                       \\ \hline
\end{tabular}
\caption{Correlation of the output of various regression models used to predict log(retweet) rates given the Tweet's textual content on the three types of tweet propagation: Base API reposts (Retweets), Base reposts plus similar messages (Combined) and Combined with spam removed (No Spam).}
\label{tab:keywordregressionsupplemental}
\end{table}

\begin{table}
\centering
\begin{tabular}{l|r|r|r}
Model & Retweets & Combined & No Spam \\ \hline
Followers & 0.3963  & 0.4151 & 0.4255\\
User Type & 0.2466 & 0.2420 & 0.2503\\
Keyword & 0.1605 & 0.1473 & 0.1719 \\
Emotion & 0.02339 & 0.02119 & 0.02279 \\ \hline
\end{tabular}
\caption{The correlation coefficient of models to predict the propagation count from messages in the H7N9 dataset. }
\label{tab:modelcor3}
\end{table}

%\begin{table}
%\begin{tabular}{l|r|r|r}
%Model & Retweets & Combined & No Spam \\ \hline
%Followers & & & \\
%User Type & & & \\
%Keyword & 0.4191 & 0.4928 & 0.4892 \\
%Emotion & 0.4257 & & \\ \hline
%\end{tabular}
%\label{tab:modelerror3}
%\caption{The mean absolute error of models to predict the propagation count from messages in the H7N9 dataset.}
%\end{table}